\documentclass{resume} % Use the custom resume.cls style



\usepackage[square,sort,comma,numbers]{natbib} % refer to https://tex.stackexchange.com/questions/54480/package-natbib-error-bibliography-not-compatible-with-author-year-citations to resolve the error...
\usepackage{bibentry}
\usepackage{xeCJK}

\usepackage[left=0.4 in,top=0.4in,right=0.4 in,bottom=0.4in]{geometry} % Document margins
\newcommand{\tab}[1]{\hspace{.2667\textwidth}\rlap{#1}} 
\newcommand{\itab}[1]{\hspace{0em}\rlap{#1}}
\name{苏隽岩} % Your name
% You can merge both of these into a single line, if you do not have a website.
%\address{+1(123) 456-7890 \\ San Francisco, CA} 
\address{Email: \href{mailto:junyan.su@my.cityu.edu.hk}{junyan.su@my.cityu.edu.hk} }
% \\ \href{https://linkedin.com/in/linkedinURL}{linkedin.com/in/linkedinURL}   %
\address{个人主页: \href{https://sujunyan.github.io}{sujunyan.github.io}}


% \usepackage[backend=bibtex]{biblatex}
% \addbibresource{\jobname.bib}
\begin{document}

\bibliographystyle{abbrv1}
\nobibliography{../../../_bibliography/underreview, ../../../_bibliography/papers}
% \nobibliography{../../../_bibliography/papers.bib}

%----------------------------------------------------------------------------------------
%----------------------------------------------------------------------------------------
%	EDUCATION SECTION
%----------------------------------------------------------------------------------------


\begin{rSection}{教育经历}

\begin{itemize}
    \item {香港城市大学}, 数据科学博士 \hfill {2020-2025}
    \item {上海科技大学}, 计算机科学与技术学士 \hfill {2015-2019} 
\end{itemize}
% 博士, 数据科学 \\

\end{rSection}

% \begin{rSection}{研究方向}
%     智慧交通系统,主要从控制与优化的角度进行研究。同时,我也对能源系统的计算方法有广泛的兴趣。
% \end{rSection}
% My research interests are intelligent transportation systems, from the perspective of control and optimization. I also have broad interests in computing methods for energy systems.

\def\FormatName#1{%
    \def\myname{Junyan Su}%
    \edef\name{#1}%
    \ifx\name\myname
      \underline{#1}%
    \else
       #1%
    \fi
}

\begin{rSection}{项目与技术栈}
    \begin{itemize}
        \item 主导开发了\href{https://www.e2pilots.com/}{\textcolor{blue}{E2Pilot}},一款专为重卡长途运输设计的节能导航系统。系统拥有网页端和移动端。
        用户只需输入起点、目的地及取送货时间窗,系统将规划最经济的路线和车速,并通过用户当前位置提供实时车速指引,以保障准时送达的前提下实现燃油成本节约。
        \item 主导开发了\href{https://github.com/sujunyan/ParExMPC}{\textcolor{blue}{ParExMPC}}。ParExMPC 是一款模型预测控制(MPC)设计工具箱。提供基于 MATLAB 的用户界面和定制的 C 代码求解器。给定一个非线性系统模型和一个优化目标,ParExMPC 可以生成一个近似最优的 MPC 控制器,并生成C代码,以方便用户部署在小型嵌入式系统中。
        \item 参与了共同署名的所有论文的代码实现以及仿真工作。 
        \item 编程语言:熟练使用 Julia、Python、C/C++、MATLAB。
    \end{itemize}
    % \textbf{Main developer} of the \href{https://www.e2pilots.com/}{\textcolor{blue}{E2Pilot}}, a navigation platform for energy-efficient long-haul timely truck transportation. \\
    % \textbf{Main developer} of the \href{https://github.com/sujunyan/ParExMPC}{\textcolor{blue}{ParExMPC}}, an open-source toolbox for real-time close-to-optimal Model Predictive Control (MPC) design providing MATLAB-based user interface and tailored C-code solver. \\
    % \textbf{Main contributor} of the simulation for ALL the publication I co-authored. \\
    % \textbf{Programming languages:} working knowledge of Julia, Python, C/C++, MATLAB.
\end{rSection}


% \begin{rSection}{Working Papers}
%     \begin{enumerate}
%          \item \bibentry{su2025maximizing}.
%          \item \bibentry{lin2025optimal}.
%     \end{enumerate}
%          % \item \bibentry{su2025maximizing}.
% \end{rSection}

\begin{rSection}{奖项与荣誉}
    \begin{itemize}
        \item 美团无人机大赛第二名, 2023
        % \item CDC学生旅行资助与研讨会支持, 2023. 
        % \item 香港城市大学研究生学费奖学金, 2023. 
        \item 香港城市大学杰出学术表现奖, 2023
        \item ACM e-Energy 最佳论文奖, 2023
        \item HK Tech 300 \& HKTSP 种子基金获得者, 2022
        \item 上海科技大学优秀毕业生, 2019
    \end{itemize}
    
    % Second Place, Meituan UAV Competition, 2023. \\
    % CDC Student Travel Grant \& Workshop Support, 2023. \\
    % Research Tuition Scholarship, City University of Hong Kong, 2023. \\
    % Outstanding Academic Performance Award, City University of Hong Kong, 2023. \\
    % ACM e-Energy Best Paper Award, 2023. \\
    % HK Tech 300 \& HKTSP Seed Fund, 2022. \\
    % Outstanding Graduate, ShanghaiTech University, 2019. 
    % Outstanding Student, ShanghaiTech University, 2016,2017,2018.
\end{rSection}

% \newpage



% \newpage

% \begin{rSection}{Presentations}
%     \begin{itemize}
%         \item ``E2Pilot: A Navigation Platform for Energy-Efficient Timely Transportation of Long-Haul Heavy-Duty Trucks'', Prototypes for Humanity, Dubai, November 2024.
%         \item ``Minimizing Carbon Footprint for Timely E-Truck Transportation: Hardness and Approximation Algorithm'', 
%         CDC 2023, Singpore, December 2023.
%         \item ``Follow the Sun and Go with the Wind: Carbon Footprint Optimized Timely E-Truck Transportation'', ACM e-Energy 2023, Orlando, Florida, June 2023.
%     \end{itemize}
% \end{rSection}

\begin{rSection}{期刊论文}
    % $^*:$ Co-primary author. 
    \begin{enumerate}
        \item \bibentry{lin2025optimala}.
        \item \bibentry{jiang2025fast}.
        \item \bibentry{su2024minimizing}.
        \item \bibentry{jiang2020distributed}.
    \end{enumerate}
\end{rSection}

\begin{rSection}{会议论文}
    % \bibliographystyle{mybibstyle}
    %% Refer to the following link to highlight my own name.
    %% https://tex.stackexchange.com/questions/33330/make-one-authors-name-bold-every-time-it-shows-up-in-the-bibliography/33379#33379
    
    % \nobibliography{papers.bib}
    \begin{enumerate}
         \item \bibentry{lin2024competitive}.
         \item \bibentry{su2023minimize}.
         \item \bibentry{su2023follow}. \textbf{Best Paper Award}.
    % \begin{enumerate}
    %     \item \bibentry{su2023follow} \textbf{(Best Paper Award)}
    %     \item \bibentry{jiang2020distributed}
    % \end{enumerate}
    % \textbf{Other Publication:}
        \item \bibentry{lin2022competitive}.
        % \item \bibentry{su2022e2pilot}.
        % \item \bibentry{su2021energy}.
        \item \bibentry{su2020distributed}.
        \item \bibentry{gao2020efficient}.
        % \item \bibentry{su2019interval}.
    \end{enumerate}
\end{rSection}

\begin{rSection}{专利}
    \begin{itemize}
        \item M. Chen., \underline{J. Su}, and Q. Lin, "Carbon Footprint Optimized Timely E-Truck Transportation", 8 Feb 2024, (Filed) U.S. Patent Application No. 18/436,350.
        % Carbon Footprint Optimized Timely E-Truck Transportation CHEN, M., SU, J. & LIN, Q., 8 Feb 2024, (Filed) Priority No. 18/436,350 Research output: Patents, Agreements and Assignments › RGC 51 - Patents (CityU)
    \end{itemize}
\end{rSection}


%----------------------------------------------------------------------------------------
% TECHINICAL STRENGTHS	
%----------------------------------------------------------------------------------------
% \begin{rSection}{SKILLS}
% \begin{tabular}{ @{} >{\bfseries}l @{\hspace{6ex}} l }
% Progamming Languages & Julia, Python, C/C++, MATLAB \\
% % Frameworks & React, Redux, Node.js, Express, Django, Mocha \\
% % Tools & Git, Docker, TravisCI, Kubernetes, AWS\\
% % Soft Skills & Time Management, Teamwork, Communication, Problem Solving, Leadership, Accountability
% \end{tabular}\\
% \end{rSection}


%----------------------------------------------------------------------------------------
% Projects
%----------------------------------------------------------------------------------------
%\begin{rSection}{PROJECTS}
%\vspace{-1.25em}
%\item \textbf{Project 1} {Language 1, Framework 1, Database, Language 2, Framework 2, DevOps Tooling} \hfill \href{www.github.com/GITHUBURL}{GitHub}
%\begin{itemize}
%    \itemsep -3pt {} 
%     \item Created a XYZ feature to accomplish ABC.
%     \item Retrieved data from XYZ to for ABC.
%    \item Implemented XYZ library for ABC.
%    \item Utilized XYZ that increased A by B\%.
% \end{itemize}
%\item \textbf{Project 2} {Language 1, Framework 1, Database, Language 2, Framework 2, DevOps Tooling} \hfill \href{www.github.com/GITHUBURL}{GitHub}
%\begin{itemize}
%    \itemsep -3pt {} 
%     \item Created a XYZ feature to accomplish ABC.
%     \item Retrieved data from XYZ to for ABC.
%    \item Implemented XYZ library for ABC.
%    \item Utilized XYZ that increased A by B\%.
% \end{itemize}
%\item \textbf{Project 3} {Language 1, Framework 1, Database, Language 2, Framework 2, DevOps Tooling} \hfill \href{www.github.com/GITHUBURL}{GitHub}
%\begin{itemize}
%    \itemsep -3pt {} 
%     \item Created a XYZ feature to accomplish ABC.
%     \item Retrieved data from XYZ to for ABC.
%    \item Implemented XYZ library for ABC.
%    \item Utilized XYZ that increased A by B\%.
% \end{itemize}
%\end{rSection} 

%----------------------------------------------------------------------------------------
% \begin{rSection}{Extra-Curricular Activities} 
% \begin{itemize}
%     \item 	Sample bullet point.
%     \item	Sample bullet point.
% \end{itemize}
% 
% 
% \end{rSection}

%------------------------
% Use this more detailed section if you have Relevant work experience
% keep your resume to 1 page, if you need to remove a project to display relevant experience
% that is okay
% ----------------------------
% \begin{rSection}{EXPERIENCE}

% \textbf{Role Name} \hfill Jan 2017 - Jan 2019\\
% Company Name \hfill \textit{San Francisco, CA}
%  \begin{itemize}
%     \itemsep -3pt {} 
%      \item Achieved X\% growth for XYZ using A, B, and C skills.
%      \item Led XYZ which led to X\% of improvement in ABC
%     \item Developed XYZ that did A, B, and C using X, Y, and Z. 
%  \end{itemize}
 
% \textbf{Role Name} \hfill Jan 2017 - Jan 2019\\
% Company Name \hfill \textit{San Francisco, CA}
%  \begin{itemize}
%     \itemsep -3pt {} 
%      \item Achieved X\% growth for XYZ using A, B, and C skills.
%      \item Led XYZ which led to X\% of improvement in ABC
%     \item Developed XYZ that did A, B, and C using X, Y, and Z. 
%  \end{itemize}

% \end{rSection} 

%\begin{rSection}{Work History}
%\vspace{-1.25em}
%\item \textbf{Job Title} {Company} \hfill Month Year - Month Year
%\item \textbf{Job Title} {Company} \hfill Month Year - Month Year
%\item \textbf{Job Title} {Company} \hfill Month Year - Month Year
%\end{rSection} 

%----------------------------------------------------------------------------------------

\end{document}